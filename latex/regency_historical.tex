\documentclass[12pt, a4paper]{article}

\usepackage{listings} 	%highlight source code
\usepackage[left=1cm,right=1cm,top=1cm,bottom=1cm,includehead,includefoot,headheight=1.7cm,footskip=1.7cm]{geometry}	%page format
\usepackage[svgnames]{xcolor}	%color package
\usepackage{graphicx}	%importing pictures
\usepackage{fancyhdr}	%adding header and footer
\usepackage{hyperref}	%link formatting
\usepackage{siunitx}	%degree symbol
\usepackage{enumitem}
\usepackage{titlesec}	%title formatting
\usepackage{amssymb}	%math symbol

%custom color
\definecolor{sfsa}{rgb}{0, 176, 80}
\definecolor{mygray}{rgb}{0.95, 0.95, 0.95}

%title formatting
\titleformat*{\section}{\LARGE\normalfont\bfseries}

%link customization
\hypersetup{
 colorlinks	= true
 urlcolor	= blue
 linkcolor	= blue
 citecolor	= red
}

%title
\title{\textbf{\huge Regency Historical}}
\author{\huge Yehezkiel}

\lstset{% general command to set parameter(s)
    basicstyle=\ttfamily, % use ttfamily for script font
    keywordstyle=\color{blue}, % underlined bold blue keywords
    identifierstyle=, % nothing happens
    commentstyle=\color{DarkGreen}, % dark green comments
    stringstyle=\color{DarkMagenta}, % typewriter type for strings
    showstringspaces=false,
    frame=single,
    breaklines=true,
    backgroundcolor=\color{mygray}
} % no special string spaces

%add SFSA logo as header
\pagestyle{fancy}
\fancyhead[R]{ % right
   \includegraphics[width=4cm]{SFSAlogo.jpg}
}


\begin{document}

%title
\maketitle
%delete page number
\thispagestyle{empty}


%next page
\newpage
%adding arabic page numbering
\pagenumbering{arabic}


%the usage of the script
\section{Introduction}
The purpose of the script is to devide the full historical data per regencies.


\section{script}
First, we need to make sure that there are no predefined variables.
\begin{lstlisting}[language=R]
 rm(list = ls(all = TRUE))
\end{lstlisting}

\vskip 0.5cm
\noindent
Then, import library
\begin{lstlisting}[language=R]
 library(stringr)
\end{lstlisting}

\vskip 0.5cm
\noindent
Declare working directory. "dirname" is a function to extract the directory of the script.
\begin{lstlisting}[language=R]
script_dir <- dirname(rstudioapi::getSourceEditorContext()$path)
\end{lstlisting}

\vskip 0.5cm
\noindent
Declare the historical data directory.
\begin{lstlisting}[language=R]
#historical data directory
historical_dir <- paste0(
  substr(script_dir, 1, unlist(gregexpr("/2. script", script_dir))),
  "\\1. historical Indonesia"
)
\end{lstlisting}

\vskip 0.5cm
\noindent
Import the pixel to village table.
\begin{lstlisting}
#import village to pixel table
reference_table <- read.csv(paste0(
  substr(script_dir, 1, unlist(gregexpr("/2. script", script_dir))),
  "\\1. historical Indonesia",
  "\\village_to_pixel_final gsmap.csv"
))
\end{lstlisting}

\vskip 0.5cm
\noindent
Load the GSMaP historical data.
\begin{lstlisting}
#load the GSMaP historical data
load(paste0(
  substr(script_dir, 1, unlist(gregexpr("/2. script", script_dir))),
  "\\1. historical Indonesia",
  "\\final_historical_indo_gsmap_daily.RData"
))
\end{lstlisting}


%new page
\newpage


\vskip 0.5cm
\noindent
Define the regency, if we want to create new product for new regencies, then this put the regency name into this variable.
\begin{lstlisting}
regencies <- c("Indramayu", "Purwakarta", "Subang", "Majalengka", "Cirebon", "Metro", "Lampung Tengah", "Lampung Selatan", "Lampung Timur", "Madiun", "Ngawi", "Nganjuk", "Malang",
               "Tulungagung", "Jember", "Jombang", "Blitar", "Kediri")
\end{lstlisting}

\vskip 0.5cm
\noindent
Check whether every regencies is in the table
\begin{lstlisting}
'%!in%' <- Negate("%in%")
for(i in regencies){
  if (i %!in% reference_table$regency){
    print(i)
  }
}
\end{lstlisting}

\vskip 0.5cm
\noindent
For loop to filter the table.
\begin{lstlisting}
for(i in regencies){
  regency_table <- reference_table[which(reference_table$regency == i),]
  pixel_regency <- unique(regency_table$pixel_name)
  
  hist_regency <- hist_data_gsmap_Indo_daily[,which(colnames(hist_data_gsmap_Indo_daily) %in% pixel_regency)]
  hist_regency <- cbind(hist_data_gsmap_Indo_daily$date, hist_regency)
  colnames(hist_regency)[1] <- "date"
  
  write.csv(hist_regency, paste0(historical_dir,"\\final_historical_", i,"_gsmap.csv"), row.names = FALSE)
  write.csv(regency_table, paste0(historical_dir, "\\regency_table_", i,"_gsmap.csv"), row.names = FALSE)
}
\end{lstlisting}











\end{document}